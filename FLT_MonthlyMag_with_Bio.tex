\documentclass{article}
\usepackage{maa-monthly}

%% IF YOU HAVE FONTS INSTALLED
% \usepackage{mtpro2}
% \usepackage{mathtime}

% \theoremstyle{theorem}
% \newtheorem{theorem}{Theorem}
% \newtheorem{proposition}[theorem]{Proposition}
% \newtheorem{lemma}[theorem]{Lemma}
% \newtheorem{corollary}[theorem]{Corollary}

% \theoremstyle{definition}
% \newtheorem*{definition}{Definition}
% \newtheorem*{remark}{Remark}

\begin{document}

% On an identity leading to the Fermat's Last Theorem in a short computation

\title{On an identity leading to the Fermat's Last Theorem in a short computation}
% \title{About an identity from which the Fermat's Last Theorem reveals itself}

\markright{}
\author{Pawel Jan Piskorz}

\maketitle


\begin{abstract}
% Abstracts for articles or notes should entice the prospective reader into exploring the subject of the paper and should make it clear to the reader why this paper is interesting and important.  The abstract should highlight the concepts of the paper rather than summarize the mechanics.  The abstract is the first impression of the paper, not a technical summary of the paper. Be brief and avoid using mathematical notation except where absolutely necessary, since this brief synopsis will be used by search engines to identify your article!
We propose a procedure which allows in a direct computation to determine the possible
natural numbers as the exponents in the equation of the Fermat's Last Theorem. 
We arrive at an identity from which the Fermat's Last Theorem reveals itself.  
\end{abstract}


\noindent
%The \textit{American Mathematical Monthly} style incorporates the following \LaTeX\ packages.  These styles should %\textit{not} be included in the document header.
% \begin{itemize}
% \item times
% \item pifont
% \item graphicx (this package is included in the {\sc Monthly} \TeX\ style file and might cause errors or conflicts when compiling your document.  If you remove it, it should compile just fine.)
% \item color
% \item AMS styles: amsmath, amsthm, amsfonts, amssymb
% \item url
% \end{itemize}
% Use of other \LaTeX\ packages should be minimized as much as possible. Math notation, like $c = \sqrt{a^2 +b^2}$, can be left in \TeX's default Computer Modern typefaces for manuscript preparation; or, if you have the appropriate fonts installed, the \texttt{mathtime} or \texttt{mtpro} packages may be used, which will better approximate the finished article.

% Web links can be embedded using the \verb~\url{...}~ command, which will result in something like \url{http://www.maa.org}.  These links will be active and stylized in the online publication.

% \section{First-level section heading.}


\section{Introduction.}

We write the equation
\begin{equation}
\label{eq:Fermat}
X^N + Y^N = Z^N
\end{equation}
for which
$X, Y, Z \in \mathbb{Z_{+}}$ are integers
greater than zero
and $N \in \mathbb{N}$ is a natural number.

Without loss of generality we rewrite Equation~(\ref{eq:Fermat}) as
\begin{equation}
\label{eq:FermatEquation}
(x + Mpx)^N + Y^N p = Z^N p
\end{equation}
with $X=x + Mpx$.
We have introduced a real number parameter $p \in \mathbb{R}$ the value of which will be later set to 1\@.
The number $M \in \mathbb{N}$ is a natural number.
With such assumptions we must have $x \in \{ \frac{1}{M+1}, \frac{2}{M+1}, \frac{3}{M+1}, \ldots \}$
in order for the variable $X$ to assume integer values.
E.g.: if $M=1$, $p=1$ then $x \in \{ \frac{1}{2}, \frac{2}{2}, \frac{3}{2}, \ldots \}$
and then $X \in \{1, 2, 3, \ldots \}$\@.
We always arrive at $X \in \{1, 2, 3, \ldots \}$ for any natural number $M$ and $p=1$,
i.e.~we have ensured that always $X \in \{1, 2, 3, \ldots \}$ no matter which natural number $M$ we take
into account. Similarly we can ensure that $Y$ and $Z$ are also integers greater than zero.


\section{Computations.}

We take partial derivative of both sides of Equation~(\ref{eq:FermatEquation}) with respect to $p$ 
\begin{equation}
\label{eq:FermatEquationTakingDerivative}
\frac{\partial (x + Mpx)^N}{\partial p} + \frac{\partial Y^N p}{\partial p} = \frac{\partial Z^N p}{\partial p}
\end{equation}
obtaining a new equation
\begin{equation}
\label{eq:FermatEquationDerivative}
N M (x + Mpx)^{N-1} x + Y^N  = Z^N
\end{equation}
Now we set $p=1$ in the Equation~(\ref{eq:FermatEquationDerivative}) receiving
\begin{equation}
\label{eq:AfterPartialDerivativeWith_p_as_one}
N M (M+1)^{N-1} x^N + Y^N  = Z^N
\end{equation}
If we set $p=1$ in Equation~(\ref{eq:FermatEquation}) on the other hand we obtain
\begin{equation}
\label{eq:two_to_compare}
(M+1)^{N} x^{N} + Y^N  = Z^N
\end{equation}
We compare the coefficients at the term with $x^{N}$ in Equations~(\ref{eq:AfterPartialDerivativeWith_p_as_one}) 
and~(\ref{eq:two_to_compare})
receiving a constraint equation for $N$
\begin{equation}
\label{eq:constraint_for_N}
NM(M+1)^{N}/(M+1) = (M+1)^{N}
\end{equation}
and therefrom we obtain the values of exponent $N$ as a function of $M$
\begin{equation}
\label{eq:constraintNM}
N = \frac{M+1}{M}
\end{equation}
We find the values of $N(M)$ as
\begin{eqnarray}
\label{eq:constraint}
N(M=1) = \frac{2}{1} = 2    \\ \nonumber
N(M=2) = \frac{3}{2}        \\ \nonumber
N(M=3) = \frac{4}{3}        \\ \nonumber
N(M=4) = \frac{5}{4}        \\ \nonumber
\vdots                      \\ \nonumber
N(M=\infty) = 1             \\ \nonumber
\end{eqnarray}
We can state that with our assumption of having $N \in \mathbb{N}$ we have to reject
all $N(M)$ solutions above which are not natural numbers.

\section{Conclusion.}

We have started with the Equation~(\ref{eq:Fermat}) with assumption that $X, Y, Z$ are positive 
integers and $N$ is a natural number.
We see from our computations
that this equation is valid for natural exponents only if the exponents in it are either 
$N=1$ or $N=2$ what is in agreement with the Fermat's Last Theorem.



% \begin{theorem}[Pythagorean Theorem]
% Theorems, lemmas, axioms, and the like are stylized using italicized text. These environments can be numbered or unnumbered, at the author's discretion.
% \end{theorem}

% \begin{proof}
% Proofs set in roman (upright) text, and conclude with an ``end of proof'' (q.e.d.) symbol that is set automatically when you end the proof environment.  When the proof ends with an equation or other non-text element, you need to add \verb~\qedhere~ to the element to set the end of proof symbol; see the \texttt{amsthm} package documentation for more details.
% \end{proof}

% \begin{definition}[Secant Line]
% Definitions, remarks, and notation are stylized as roman text.  They are typically unnumbered, but there are no hard-and-fast rules about numbering.
% \end{definition}

% \begin{remark}
% Remarks stylize the same as definitions.
% \end{remark}


\begin{acknowledgment}{Acknowledgment.}
This article is in honor of American mathematician Kenneth S.~Miller. Due to his technique of computing the expected value and the standard deviation of number of successes in Bernoulli 
trials   %~\cite{kn:Miller_K_Book} 
we were able to obtain our results. 
Author would also like to thank an
anonymous student without whom it would take longer to
write this paper and who suggested placing the number $p$ next to symbols $Y^{N}$ and $Z^{N}$
in Equation~(\ref{eq:FermatEquation}) while the author was working on the universal expression 
of the natural number $X$ using rational number $x$ and a real parameter $p$ for partial differentiation.
\end{acknowledgment}

\begin{thebibliography}{2}
% \bibitem{hopkins} Hopkins, B. ed. (2009). \textit{Resources for Teaching Discrete Mathematics.} Washington DC: Mathematical Association of America.

% \bibitem{parker13} Parker, A. (2013). Who solved the Bernoulli equation and how did they do it? \textit{Coll. Math. J.} 44(2): 89--97. doi.org/10.4169/college.math.j.44.2.089.

% \bibitem{kn:Miller_K_Book} Miller, Kenneth~S.\ 
%                                           (1956) 
%                                           {\em Engineering Mathematics\/} Dover Publications, Inc., New York

% \bibitem{wikipedia} \url{https://en.wikipedia.org} \ Fermat's Last Theorem
\bibitem{wikipedia} \url{https://en.wikipedia.org/wiki/Fermat%27s_Last_Theorem} \ Fermat's Last Theorem




\end{thebibliography}

\begin{biog}
\item[Pawel Jan Piskorz] received his Ph.D.~in chemical sciences from Jagiellonian University in Krakow, Poland. 
He worked as postdoctoral researcher in Supercomputer Computations Research Institute in Florida State University
Computational Chemistry Group and as Computer Programmer-Analyst in the State of Florida Department
of Environmental Protection. He enjoys working on problems in mathematics.
\begin{affil}
Krakowska 55, 31-066 Krakow, Poland\\
pjpxyz@protonmail.com
\end{affil}
\end{biog}
\vfill\eject

\end{document}
