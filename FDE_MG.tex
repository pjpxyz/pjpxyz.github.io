% \documentclass[oupthm]{CUP-JNL-BCM}%
\documentclass[12pt]{article}
% \usepackage{amssymb}

%%%% Packages
\usepackage{graphicx}
\usepackage{multicol,multirow}
\usepackage{amsmath,amssymb,amsfonts}
\usepackage{mathrsfs}
%%\usepackage{amsthm}
\usepackage{rotating}
\usepackage{appendix}
\usepackage[numbers]{natbib}
\RequirePackage{hyperref}
\usepackage{hyperref}

% \theoremstyle{oupplain}
% \newtheorem{theorem}{Theorem}[section]
% \newtheorem{lemma}[theorem]{Lemma}
% \theoremstyle{oupdefinition}
% \newtheorem{definition}{Definition}[section]
% \theoremstyle{oupremark}
% \newtheorem{remark}[theorem]{Remark}
% \newtheorem{example}[theorem]{Example}
% \theoremstyle{oupproof}
% \newtheorem{proof}{Proof}

\numberwithin{equation}{section}


% \articletype{RESEARCH ARTICLE}
% \jname{Canadian Mathematical Society}
%\artid{20}
% \jyear{2020}
%\jvol{4}
%\jissue{1}
%\jdoi{2190-8567}
%\raggedbottom


\begin{document}

% \begin{Frontmatter}

\title{On Fermat's Diophantine equation and its only allowed natural exponents}


% \title[Newton Complementary Duals of \textup{f}-Ideals]{Newton complementary duals of $f$-ideals\thanks{Sample text for title footnote.}}




\author{Pawel Jan Piskorz}

\maketitle


% \author{Adam Van Tuyl}

% \authormark{P.J.~Piskorz}


% \address{\orgname{DPS}, \orgaddress{\street{Krakowska 55}, \state{31-066 Krakow, Poland}} \email{paweljanpiskorz@gmail.com}}



% \address{\orgname{DPS}, \orgaddress{\street{Hamilton}, \state{ON}, \postcode{L8S 4L8}}\email{budds2@mcmaster.ca}, \email{sjbudd3@gmail.com}, \email{vantuyl@math.mcmaster.ca}}


% \keywords[2020 Mathematics Subject Classification]{11D41}

% \keywords{Fermat's Diophantine equation, Fermat's Last Theorem}

% \abstract{A square-free monomial ideal $I$ of $k[x_1,\ldots,x_n]$ is
% said to be an $f$-ideal if the facet complex and non-face complex
% associated with $I$ have the same $f$-vector.    We show that
% $I$ is an $f$-ideal if and only if its Newton complementary dual
% $\hat{I}$ is also an $f$-ideal.  Because of this duality,
% previous results about some classes of $f$-ideals can be extended
% to a much larger class of $f$-ideals.   An interesting by-product of our
% work is an alternative formulation of the Kruskal-Katona theorem
% for $f$-vectors of simplicial complexes.}


\abstract{We propose a procedure which allows to compute the only acceptable natural
exponents of the positive integers $X, Y, Z$
in the Fermat's Diophantine equation. 
We use the approach similar to the one applied in
computing of the expected value and the standard deviation of number of successes in Bernoulli 
trials presented by Kenneth S.~Miller.}




% \end{Frontmatter}


\section{Introduction}\label{sec:intro}

We write the equation from the Fermat's Last Theorem~\cite{FLT_Wikipedia}
\begin{equation}
\label{eq:Fermat}
X^N + Y^N = Z^N
\end{equation}
in which
$X, Y, Z \in \mathbb{Z_{+}}$ are integers
greater than zero
and $N \in \mathbb{N}$ is a natural number.

Without loss of generality we rewrite Equation~(\ref{eq:Fermat}) as
\begin{equation}
  \label{eq:FermatGeneralEquation}
  (x + Mpx)^{N} qr + (y + Lqy)^{N} rp = (z + Krz)^{N} pq
  \end{equation}
with $p, q, r \in \mathbb{R}$
and $M, L, K \in \mathbb{N}$\@.
Now we set $q=1$ and $r=1$
with $Y = y + Ly$ and $Z = z + Kz$
obtaining
\begin{equation}
\label{eq:FermatEquation}
(x + Mpx)^N + Y^N p = Z^N p
\end{equation}
with $X=x + Mpx$.
We have introduced a real number parameter $p \in \mathbb{R}$ the value of which will 
be later set to 1\@.
The number $M \in \mathbb{N}$ is a natural number.
With such assumptions we must have $x \in \{ \frac{1}{M+1}, \frac{2}{M+1}, \frac{3}{M+1}, \ldots \}$
in order for the variable $X$ to assume integer values.

% E.g.:

% if $M=1$, $p=1$, $x \in \{ \frac{1}{2}, \frac{2}{2}, \frac{3}{2}, \ldots \}$
%  then $X = \{\frac{1}{2} + 1 \cdot \frac{1}{2}, \frac{2}{2} + 1 \cdot \frac{2}{2}, \frac{3}{2} + 1 \cdot \frac{3}{2}, \ldots \}$
% what gives 
% $X \in \{1, 2, 3, \ldots \}$;

% if $M=7$, $p=1$, $x \in \{ \frac{1}{8}, \frac{2}{8}, \frac{3}{8}, \ldots \}$
%  then $X = \{\frac{1}{8} + 7 \cdot \frac{1}{8}, \frac{2}{8} + 7 \cdot \frac{2}{8}, \frac{3}{8} + 7 \cdot \frac{3}{8}, \ldots \}$
% what gives again
% $X \in \{1, 2, 3, \ldots \}$\@.

We always arrive at $X \in \{1, 2, 3, \ldots \}$ for any natural number $M$ and $p=1$,
i.e.~we have ensured that always $X \in \{1, 2, 3, \ldots \}$ no matter which natural number $M$ we take
into account. It is because for $p=1$, $X = x + Mx = x(M+1)$\@.

Similarly we can ensure that $Y$ and $Z$ are also integers greater than zero
appropriately choosing the natural number parameters $L$ and $K$ and appropriately real numbers $q$ and~$r$\@.



\section{Computations}\label{sec:computations}

We take partial derivative of both sides of Equation~(\ref{eq:FermatEquation}) with respect to $p$ 
\begin{equation}
\label{eq:FermatEquationTakingDerivative}
\frac{\partial (x + Mpx)^N}{\partial p} + \frac{\partial Y^N p}{\partial p} = \frac{\partial Z^N p}{\partial p}
\end{equation}
We compute the partial derivtive of $(x + Mpx)^N$ as follows
\begin{eqnarray}
\frac{\partial (x + Mpx)^N}{\partial p}
=
N(x + Mpx)^{N-1}  \frac{\partial (x + Mpx)}{\partial p}   \\ \nonumber
=
N(x + Mpx)^{N-1} Mx 
=
NM  (Mp+1)^{N-1} x^{N-1}  x     \\  \nonumber
=
NM  (Mp+1)^{N-1}  x^N
\end{eqnarray}
% If we set p=1 then we receive
We receive a new equation
\begin{equation}
NM  (Mp+1)^{N-1}  x^N + Y^{N} = Z^{N}
\end{equation}
in which we can set the prameter $p=1$ obtaining
\begin{equation}
\label{eq:AfterPartialDerivativeWith_p_as_one}
NM  (M+1)^{N-1}  x^N + Y^{N} = Z^{N}
\end{equation}
If we set $p=1$ in Equation~(\ref{eq:FermatEquation}) on the other hand we obtain
\begin{equation}
\label{eq:two_to_compare}
(M+1)^{N} x^{N} + Y^N  = Z^N
\end{equation}
We compare the coefficients at the term with $x^{N}$ in Equations~(\ref{eq:AfterPartialDerivativeWith_p_as_one}) 
and~(\ref{eq:two_to_compare})
receiving a constraint equation for $N$
\begin{equation}
\label{eq:constraint_for_N}
NM(M+1)^{N}/(M+1) = (M+1)^{N}
\end{equation}
and therefrom we obtain the values of exponent $N$ as a function of $M$
\begin{equation}
  \label{eq:constraintNM}
  N(M) = \frac{M+1}{M}
\end{equation}
Quite similarly we can arrive at the formulas for $N$ as a function of $L$
\begin{equation}
  \label{eq:constraintNL}
  N(L) = \frac{L+1}{L}
\end{equation}
and for $N$ as a function of $K$
\begin{equation}
   \label{eq:constraintNK}
  N(K) = \frac{K+1}{K}
\end{equation}
% what gives the constraint equations for the functions $N(L)$ and $N(K)$ similar to the constraints in Equations~\ref{eq:constraint}
% for the function $N(M)$\@.



\section{Conclusion}\label{sec:conclusion}

We have started with the Equation~(\ref{eq:Fermat}) with assumption that $X, Y, Z$ are positive 
integers and $N$ is a natural number.
We received the system of three constraint E\-qu\-a\-tions~(\ref{eq:constraintNM}), (\ref{eq:constraintNL}) and (\ref{eq:constraintNK})
for the four unknowns $M, L, K$ and the number $N$\@.
It means that one unknown among the four ones must assume two integer values. We see below that it is the variable $N$\@.

For $N(M)$ we have
\begin{eqnarray}
\label{eq:constraint}
N(M=1) = \frac{2}{1} = 2    \\ \nonumber
N(M=2) = \frac{3}{2}        \\ \nonumber
N(M=3) = \frac{4}{3}        \\ \nonumber
N(M=4) = \frac{5}{4}        \\ \nonumber
\vdots                      \\ \nonumber
N(M=\infty) = 1             \\ \nonumber
\end{eqnarray}
% We need to accept only the natural number solutions $N(M)=1$ and $N(M)=2$ according to our assumptions.
% Similar computations lead to the expressions for $N(L)$ and $N(K)$\@. 

% In those results we have to reject the solutions which are not natural numbers.

Quite similarly we can compute $N(L)$ and $N(K)$\@.

We can state that with our assumption of having $N \in \mathbb{N}$ we have to reject
all $N(M)$, $N(L)$ and $N(K)$ solutions which are not natural numbers.
Our fourth unknown is $N$ equal to either $1$ or $2$ what is in perfect agreement with the Fermat's Last Theorem~\cite{FLT_Wikipedia}\@.


\section{Acknowledgement}\label{sec:acknowledgement}

    This article is in honor of American mathematician Kenneth~S.\ Mil\-ler. Due to his 
    technique of computing the expected value and the standard deviation of number of 
    successes in Bernoulli trials~\cite{book_by_Miller}
    we were able to obtain our results. 
    Author would also like to thank an
    anonymous student without whom it would take longer to
    write this paper and who suggested placing the number $p$ next to symbols $Y^{N}$ and $Z^{N}$
    in Equation~(\ref{eq:FermatEquation}) while the author was working on the universal expression 
    of the natural number $X$ using rational number $x$ and a real parameter $p$ for partial differentiation. 
    Author received funds from PFRON allowing purchase of a laptop on which this publication has been written.   



%%%%%%%%%%%%%%%%%%%%%%%%%%%%%%%%%%%%%%%%%%%%%%%%%%%%%%%%%%%%%%%%%%%%%%%%%%%%%%%%%%%%%%%%%%%%%%%%%%%%%%%%%%%%%%%%%%%%%%%%%%%%%%


% \begin{Backmatter}


\begin{thebibliography}{9}


    \bibitem{FLT_Wikipedia} % K. Ansaldi, K. Lin, and Y. Shen,
    Fermat's Last Theorem. \newline
    \url{https://en.wikipedia.org/wiki/Fermat%27s_Last_Theorem}
    % Preprint (2017). 
    % \url{https://en.wikipedia.org/wiki/Fermat%27s_Last_Theorem}

\bibitem{book_by_Miller}
Kenneth S.~Miller, \textit{Engineering Mathematics} (1956), Dover Publications, Inc., New York.



% \bibitem{bib1}
% S. Faridi, The facet ideal of a simplicial complex. \textit{Manuscripta Math.} 109 (2002), 159--174.

% \bibitem{bib2} G.Q. Abbasi, S. Ahmad, I. Anwar, and W.A. Baig, \textit{$f$-Ideals of degree 2. Algebra Colloq}. 19 (2012), 921--926.

% \bibitem{bib3}
% I. Anwar, H. Mahmood, M.A. Binyamin, and M.K. Zafar, On the characterization of $f$-ideals. \textit{Comm.
% Algebra} 42 (2014), 3736--3741.


% \bibitem{bib4}
% J. Guo and T. Wu, On the $(n,d)^{th}$ $f$-Ideals. \textit{J. Korean Math. Soc.} {52} (2015), 685--697.


% \bibitem{bib5}
% J. Guo, T.Wu, and Q. Liu, $F$-Ideals and $f$-Graphs. \textit{Comm. Algebra} 45 (2016), 3207--3220.


% \bibitem{bib6}
% H. Mahmood, I. Anwar, and M.K., Zafar, A construction of Cohen-Macaulay $f$-graphs.
% \textit{J. Algebra Appl.} {13} (2014), 1450012, 7 pp.

% \bibitem{bib7}
% H. Mahmood, I. Anwar, M.A. Binyamin, and S. Yasmeen, On the connectedness of $f$-simplicial complexes.
% \textit{J. Algebra Appl.} {16} (2017), 1750017, 9 pp.

% \bibitem{bib8} K. Ansaldi, K. Lin, and Y. Shen,
% Generalized Newton complementary duals of monomial ideals.
% Preprint (2017). \url{arXiv:1702.00519v1}

% \bibitem{bib9} B. Costa and A. Simis, New constructions of Cremona maps. \textit{Math. Res. Lett.} {20} (2013), 629--645.


% \bibitem{FLT_Wikipedia} % K. Ansaldi, K. Lin, and Y. Shen,
% Fermat's Last Theorem.
% % Preprint (2017). 
% \url{https://en.wikipedia.org/wiki/Fermat%27s_Last_Theorem}




\end{thebibliography}

% \printaddress

% \end{Backmatter}

\end{document}
